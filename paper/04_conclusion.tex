\section*{Заключение}
\addcontentsline{toc}{section}{\hspace{7mm}Заключение}

В рамках данной работы достигнуты следующие результаты:

\begin{enumerate}

\item
Была изучена предметная область анализа тональности,
был дан обзор современным исследованиям и задачам этой области,
а также существующим методам решения.

\item
Было реализовано несколько популярных алгоритмов машинного обучения 
для определения тональности текста. 
Проведено исследование их работы в области
микроблогов, проанализированы их достоинства и недостатки.
В целом, наивный Байесовский классификатор при
совместном использовании юниграмм
и биграмм в качестве классификационных признаков
показал лучшие результаты, чем словарный метод определения тональности.

\item
Автором разработано веб-приложения для поиска и анализа мнений
в социальной сети Twitter. Приложение может использоваться для
повышения эффективности при анализе мнений в социальных медиа 
(например, при проведении социологических опросов или маркетинговых исследований).

\item
Был получен опыт работы с Twitter API.

\end{enumerate}

Были выявлены следующие возможности для дальнейшей работы:
\begin{enumerate}

\item
Добавление поддержки русского языка. Для этого необходимо собрать размеченную обучающую выборку (никаких изменений в код вносить не придётся).

\item
Исследование использования ансамблей моделей для определения тональности.

\item
Исследование обнаружения субъективности в тексте. Так как не все сообщения
содержат субъективную оценку, то просто классификация всех сообщений
на положительные и отрицательные будет давать неверные результаты.
Просто текстовая классификация на основе n-грамм работает с высокой точностью,
но низкой полнотой.

\item
Использование информации из хештегов может дать дополнительный прирост 
качества (в настоящий момент они удаляются, чтобы не создавать ``шум''). 
Для этого необходимо научиться корректно разбивать хештеги на слова.

\item
Автоматическая коррекция ошибок в написании слов может улучшить результаты классификации и уменьшить размерность.

\end{enumerate}

Исходный код проекта открыт.