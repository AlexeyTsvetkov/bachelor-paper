\section*{Введение}
\addcontentsline{toc}{section}{\hspace{7mm}Введение}

``Анализ тональности'' (\textit{Sentiment Analysis}) ---  класс методов анализа 
содержимого, предназначенный для автоматизированного выявления в текстах 
эмоционально окрашенной лексики и эмоциональной оценки авторов по 
отношению  к объектам, речь о которых идёт в тексте~\cite{wikisent}. 

Мнения других людей влияли на наш процесс принятия решений ещё до 
распространения интернета. Однако, если раньше было возможным узнать мнение 
лишь у ограниченного числа знакомых, то в последнее десятилетие, в связи с 
ростом популярности сети интернет, всё большее значение приобретают отзывы, 
оставляемые пользователями в интернет-магазинах, блогах, социальных сетях, а 
также специализированных ресурсах (``Яндекс.Маркет'', ``Кинопоиск''). 

Согласно опросу, проведённому компанией Dimensional Research~\cite{dimresearch}, 88\% опрошенных считают, что чтение положительных или отрицательных 
отзывов в интернете влияет на их решение при покупке товаров (что согласуется с 
данными, полученными при проведении аналогичного опроса в АиФ России~\cite{aif}). 
Отзывы на качество обслуживания также непосредственно влияют на продажи: 
продажи интернет-магазинов с рейтингом ``5 звёзд'' на ``Яндекс.Маркете'' на 55\% 
больше магазинов с рейтингом ``4 звезды''~\cite{medianation}.

Всё чаще пользователи оставляют отзывы не на специализированных 
сайтах отзывов, а в социальных сетях. 
Отзывы в социальных сетях оставляет на 10\% большее число покупателей, чем на 
специализированных сайтах-агрегаторах~\cite{dimresearch}.
Так как социальные сети содержат не только отзывы, а объём сообщений слишком 
велик для ручной обработки, актуальной является задача автоматического поиска 
и классификации отзывов.
 
``Твиттер'' (\textit{Twitter}) --- одна из самых популярных социальных сетей. 
Число активных пользователей превышает 200 миллионов, и они оставляют более 
400 миллионов сообщений в день~\cite{twitter_users}. 
Ограничение на длину сообщения (140 символов), большой набор используемой 
лексики, сленга и грамматических ошибок делают автоматический поиск и анализ
мнений нетривиальной задачей.

Методы машинного обучения применяются в задачах определения тональности 
уже долгое время~\cite{panglee}, однако их применение в сегменте микроблогов 
началось сравнительно недавно~\cite{distsuperv}.

\vspace{0.5cm}

Целью данной работы является исследование методов автоматического 
определения тональности на основе методов машинного обучения в социальной 
сети Twitter. 

Для достижения поставленной цели требуется:

\begin{enumerate}

\item Дать обзор существующим исследованиям в области анализа тональности.

\item Изучить и реализовать алгоритмы основные машинного обучения, используемые для решения задачи автоматического определения тональности.

\item Исследовать качество работы алгоритмов, выявить их достоинства и недостатки.

\item Создать веб-приложение для практического использования автоматического определения тональности в социальной сети Twitter. 

\end{enumerate}