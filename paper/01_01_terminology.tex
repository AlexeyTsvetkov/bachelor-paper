\subsection{Терминология}

 В рамках данной работы используются терминология предметной области, которую необходимо пояснить.

``{\bfАнализ тональности}'' (\textit{sentiment analysis}), ``{\bfанализ мнений}''(\textit{opinion mining}) --- названия предметной области.

{\bfОбъектом} анализа тональности может являться любая сущность, относительно которой выражается мнение. Например, это может быть продукт, сервис, организация или событие. Объект может обладать множеством \textit{компонентов} (составных частей) и \textit{атрибутов} (свойств), вместе составляющих множество {\bfаспектов} (\textit{features})~\cite{multi_faceted}.

Конкретная модель телефона --- объект. Он состоит из компонентов (\textit{экран, аккумулятор, камера}) и обладает набором атрибутов (\textit{качество передачи голоса, размер}), которые вместе составляют множество аспектов. Мнение может быть выражено, как о самом объекте, так и о любом из его аспектов.

В сообщении ``Мне так нравится новый iPhone. Качество экрана просто потрясающее!'' первое предложение выражает положительное мнение о самом объекте ``iPhone'', а второе --- о его аспекте ``экран''.

{\bfАвтор мнения} (\textit{opinion holder}) --- человек или организация, которые выражают это мнение. 

{\bfТональность мнения} (\textit{opinion orientation}) --- эмоциональная оценка, даваемая автором мнения объекту и/или любому из его аспектов, например \textit{положительная} или \textit{отрицательная} . Строго говоря, тональность может принимать большее число значений, однако чаще всего используются именно эти классы. Часто в литературе можно встретить использование значения \textit{нейтральной} тональности, однако его интерпретация неоднозначна. 
В некоторых работах~\cite{panglee} нейтральная тональность определяется, как промежуточное значение между положительной и отрицательной, а в некоторых, как отсутствие субъективной оценки. В данной работе автор работы придерживается последней интерпретации из-за трудности формализации промежуточного значения. Например, в высказывании ``посредственный экран'' оценка, даваемая автором мнения аспекту ``экран'', является скорее негативной, несмотря на то, что слово ``посредственный'' определяется в словаре, как ``заурядный, средний''~\cite{wiki_middling}.

{\bfМоделью объекта} $o$ называется конечное множество аспектов $F = \setof{f_1, f_2, \ldots , f_n}$, которое включает в себя сам объект в качестве особого аспекта.

{\bfМнения} делятся на два типа~\cite{multi_faceted}:
\begin{enumerate}

\item{
  {\bfПростое} мнение (\textit{direct opinion}). Простое мнение формально определяется, как кортеж \tuple{A, o, f, oo, t, d}, где в документе $d$ автором мнения $A$ аспекту $f$ объекта $o$ дана оценка $oo$ в момент времени $t$. Например, в высказывании ``В Ubuntu очень красивый интерфейс'' автор мнения дал положительную оценку ``красивый'' аспекту ``интерфейс'' объекта ``Ubuntu''.
}

\item{
  {\bfСравнительное} мнение(\textit{comparative opinion}). Сравнительное мнение характеризуется тем, что автор $A$ предпочёл одному или нескольким объектам из множества $O_1$ один или несколько объектов из множества $O_2$, давая сравнительную оценку $oo$ их аспекту $f$. Например, в высказывании ``Интерфейс Ubuntu проще Windows 8'' автор мнения предпочёл объект ``Ubuntu'' объекту ``Windows 8'', дав аспекту ``интерфейс'' сравнительную оценку ``проще''. 
}

\end{enumerate}

В работе также используется терминология, специфичная для Twitter.

{\bfМикроблог} (\textit{microblog}) --- аналог обычного веб-блога (интернет-дневника), с ограничением на длину сообщения (в Твиттере она составляет 140 символов). 

{\bfТвит} (\textit{tweet}) --- сообщение пользователя в Твиттере. Может включать в себя текст, упоминание другого пользователя, гиперссылки и хештэги.

{\bfРетвит} (\textit{retweet}) --- возможность скопировать сообщение другого пользователя в свою ленту (с сохранением авторства). Существует, две разновидности --- нативный (предоставляемый платформой) и текстовый (сообщение, начинающееся с ``RT @имя\_автора:'').

{\bfХештэг} (\textit{hashtag}) --- это слово или фраза, которым предшествует символ \#. Пользователи могут объединять группу сообщений по теме или типу с использованием хэштегов — слов или фраз, начинающихся с \#. В Twitter в хештэгах можно использовать латинские и кириллические буквы, цифры и знаки подчёркивания, однако часто пользователи не используют знаки подчёркивания (``\#отличныйдень''вместо ``\#отличный\_день'').

{\bfУпоминание} (\textit{mention}) --- ссылка на другого пользователя.  Начинается с символа @, после чего идёт никнейм другого пользователя, состоящий из символов латинского алфавита, цифр и знаков подчёркивания. Пользователям приходят уведомления, когда их кто-то упомянул. Иногда используется вместо хештэгов для группировки сообщений по темам (например, вместо ``\#microsoft'' используется ``@microsoft'').