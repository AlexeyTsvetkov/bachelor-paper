\subsection{Проблемы автоматического определения тональности}

Проблемы, возникающие при автоматическом определении тональности, можно 
разделить на общие и специфичные для микроблогов.

\vspace{0.5cm}

Среди общих проблем можно выделить следующие:

\begin{enumerate}

\item 
Зависимость значения тональности от предметной области. Так, в области 
фильмов слово ``непредсказуемый'' может иметь положительный оттенок, 
однако в области клиентского обслуживания это не так.
При использовании методов обучения с учителем, алгоритм классификации (
например, наивный Байесовский классификатор) сам формирует значения 
тональности из обучающей выборки, поэтому для правильной классификации 
достаточно, чтобы обучающая и тестовые выборки имели общую предметную 
область.

На практике запросы пользователей не обязательно ограничены
какой-либо одной областью, поэтому возможна категоризацая текста в два 
прохода: сначала осуществляется тематическая классификация 
документа, затем классификация тональности.

\item 
Использование отрицания может изменить тональность остальной части 
высказывания на обратную. 
Например, в высказывании ``Раньше мне \textit{очень нравились} смартфоны Nokia. Качество сборки в Lumia 800, действительно, \textit{на высоте}. \textit{Однако}, Windows Phone 7 \textit{всё портит}''. 
В первом и втором предложении автор высказывает положительное мнение, но 
из-за использования отрицания в третьем предложении общая тональность 
относительно объекта ``Nokia'' отрицательная. 

Для методов машинного обучения, использующих модель 
типа ``набор слов'' (\textit{bag-of-words}), 	в качестве простой эвристики можно 
искусственно добавлять частицу ``не'' к соседним словам (к примеру, для 
высказывания ``Мне не очень нравится камера iPhone'' получится строка 
``Мне не не\_очень не\_нравится камера iPhone''), однако это не очень точное моделирование отрицания (кроме того, отрицание может быть выражено неявно).	

Исследования в данной области находятся на очень ранней стадии~\cite{Wiegand2010}.

\item 
Использование сарказма плохо поддаётся автоматическому определению. 
Высказывания, содержащие сарказм могут иметь общую тональность, 
обратную тональности отдельных слов 
(``\textit{Отличная} книга для страдающих бессонницей!''), 
или выражать мнение в скрытой форме (вопрос ``Где я?'' 
в обзоре GPS-навигатора). В этом случае, сарказм может с трудом распознаваться даже людьми. 

В одной из последних работ в данной области авторам удалось 
добиться точности на уровне 78\% на коллекции отзывов на товары, используя 
метод частичного обучения (\textit{semi-supervised learning})~\cite{Tsur2010}.

\item 
Значение тональности зависит от того, кому необходимо провести анализ. 
Так, для компании Samsung высказывание ``У Samsung показывает отличные продажи :)'' несёт положительное значение тональности, а для компании Nokia --- наоборот.

\end{enumerate}

\vspace{0.5cm}

К специфичным для Твиттера проблемам относятся:

\begin{enumerate}

\item 
Большой словарь употребимых слов.  
Как показывают исследования~\cite{Saif2012}	, 93\% встречающихся слов 
употребляются менее, чем 10 раз (78\% на корпусе рецензий фильмов из IMDB). 
Это объясняется частым использованием сленга, намеренным и ненамеренным искажением написания слов, использованием разных регистров при написании одного и того же слова.

\item 
Малое количество слов в каждом сообщении. 
Из-за ограничения на длину сообщения, средняя длина сообщения на 
английском составляет 14 слов или 78 символов~\cite{Go2009}.

\item 
Большие объёмы данных. 
Пользователи публикуют более 400 миллионов сообщений в день, 
1\% сообщений доступны всем желающим, 
50\% сообщений доступны для компаний-партнёров~\cite{streaming}. 
Даже 1\% --- это более 40 миллионов сообщений в день, что является 
достаточно большим объёмом для обработки локально. 
Так как все сообщения можно сначала обработать независимо, 
а затем агрегировать результаты, 	разумным 
может быть применение распределённых вычислений, к примеру, 
использование парадигмы MapReduce~\cite{Lin2012, sentiment_mapreduce}.

\end{enumerate}