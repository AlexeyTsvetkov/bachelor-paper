\subsection{Подходы к автоматическому определению тональности}

Основные подходы определения тональности можно разделить на следующие категории~\cite{habrasent}:

\begin{enumerate}

\item { Подход, основанный на правилах (\textit{rule-based approach}), заключается в применении набора правил, выявленного экпертами на основе анализа предметной области. Пример такого правила:
\begin{framed}
Если высказывание содержит один или несколько положительных прилагательных из набора \{``хороший'', ``качественный'', ``свежий'' \ldots \} и не содержит прилагательных из набора \{``бездарный'', ``ужасный'', ``жуткий'' \ldots \}, то классифицировать тональность, как ``положительная''.
\end{framed}
Достоинства подхода:
\begin{itemize}
\item {  Может показывать хорошие результаты при большом наборе правил.}
\end{itemize}

Недостатки подхода:
\begin{itemize}
\item {  Создание большого набора достаточно трудозатрано.  }
\item {  Применение этого подхода для микроблогов может быть проблематично из-за ``зашумленности'' (использование сленга, неправильное написание слов). }
\end{itemize} 

Более подробно определение тональности с помощью подходов, основанных на правилах, описано в~\cite{rule_romip1, rule_romip2}.

}

\item {
  Подход, основанный на использовании словарей оценочной лексики (\textit{affective lexicons}). 
  Для каждого слова, встречаемого в документе, из словаря получают значение тональности. Чтобы получить итоговую тональность необходимо взять среднее арифметическое или вычислить сумму значений тональности всех слов из документа. 

Достоинства подхода:
\begin{itemize}
\item {  Простота использования. }
\end{itemize}

Недостатки подхода:
\begin{itemize}
\item { Не универсальность: для каждой предметной области требуется свой словарь. }
\item {  Создание такого словаря может быть проблематичным.  }
\end{itemize} 

  Для английского языка в качестве такого словаря можно использовать SentiWordNet~\cite{sentiwordnet} или ANEW~\cite{anew}.
}

\item {
  Подходы, основанные на обучении с учителем (\textit{supervised learning}). 
  Обучение с учителем --- один из разделов машинного обучения. Алгоритм классификации тренируется на основе обучающей выборке (корпусе), состоящей из документов, классы которых заранее известны. 

Достоинства подхода:
\begin{itemize}
\item {  Хорошая точность при определении тональности. }
\item {  На основе обучающей выборки классификатор может выделить 
признаки, влияющие на тональность. Таким образом, проблема зависимости от предметной области решается путём использования обучающей выборки из той же области. }
\item {  Существует множество способов улучшить точность. }
\end{itemize}

Недостатки подхода:
\begin{itemize}
\item {  Требуется размеченная обучающая выборка.  }
\item {  Результаты сильно зависят от выбранного алгоритма, его параметров, обучающей выборки. }
\end{itemize} 

}

\item {
  Подходы, основанные на обучении без учителя (\textit{unsupervised learning}). 
  Обучение без учителя является ещё одним из разделов машинного обучения. Отличие состоит в том, что в этом случае для тренировки алгоритма используется обучающая выборка, состоящей из документов, классы которых заранее неизвестны (или известны, но эта информация не используется алгоритмом).

Достоинства подхода:
\begin{itemize}
\item {  Для обучения не требуется размеченная выборка. }
\end{itemize}

Недостатки подхода:
\begin{itemize}
\item {  Точность обычно значительно ниже, чем у алгоритмов, основанных на обучении с учителем.  }
\end{itemize} 

}

\end{enumerate}

Так как подходы, основанные на обучении с учителем, показывают хорошие результаты при анализе традиционных блогов и сайтов-отзывов, далее в работе даётся их подробное описание, описываются способы представления данных и улучшения точности  работы алгоритмов, а также результаты работы в области микроблогов.