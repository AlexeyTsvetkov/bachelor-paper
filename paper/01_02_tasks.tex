\subsection{Задачи анализа тональности}

Основные задачи, решаемые в области анализа тональности, 
включают в себя~\cite{panglee}:
\begin{enumerate}


\item {
  Задача определения документов или частей документа, которые содержат в себе 
  мнение (\textit{subjectivity detection}). Для документов эта задача сводится к 
  задаче бинарной текстовой классификации на классы {\bfсубъективных} (
  содержащих мнение) и {\bfобъективных} (содержащих факты) документов.
}

\item {
  Задача определения тональности. Чаще всего встречается определение 
  тональности на уровне документа (\textit{document-level sentiment classification}). 
  Формально задача определяется так: 
  из множества $D$ документов, содержащих мнения, для каждого $d \in D$, 
  который содержит высказывание об объекте $o$, необходимо определить 
  тональность $oo$ аспекта $f$~\cite{sentsubj}. 
  Также, определение тональности может быть осуществлено на уровне 
  предложения (\textit{sentence-level sentiment classification}) или аспекта (\textit{
  feature-level sentiment classification}).

  Существующие решения делают следующиее допущение: документ $d$, 
  содержащий мнение, выражает мнение об одном объекте $o$, 
  от одного автора $A$. В общем случае автор документа может выражать мнение 
  относительно нескольких объектов, однако в случае отзывов (и особенно в 
  контексте Twitter) в большинстве случаев это допущение справедливо для 
  простых (не сравнительных) мнений.

  Учитывая допущение, эта задача сводится к бинарной текстовой классификации 
  на классы {\bfположительных} (позитивных) и {\bfотрицательных} (негативных) 
  документов.
}

\item {
  Наконец, необходимо представить информацию о мнениях в некоторой краткой 
  сводке. Это может быть:
  
  \begin{itemize}

  \item Агрегация пользовательских оценок.
  \item Идентификация групп авторов, имеющих схожие мнения.
  \item Выявление пунктов согласия и противоречия.
  \item Статистическое или текстовое обобщение документов.

  \end{itemize}
}

\end{enumerate}

Задачи 1 и 2 можно рассматривать, как задачи текстовой классификации. Решение 
этих задач можно получить с помощью иерархической бинарной классификации (
сначала документ классифицируется, как объективный или субъективный, затем 
субъективные документы --- как положительные или отрицательные) или с 
помощью классификации документа по трём классам \{положительный, 
нейтральный, отрицательный\}. 

Задача определения тональности встречается чаще всего, поэтому далее в работе 
будет подразумеваться именно она.